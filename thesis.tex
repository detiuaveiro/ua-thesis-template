%!TEX encoding = UTF-8 Unicode

%% Thesis main document

% Remove 'coverdepname' parameter to remove department name from cover pages
% Add 'final' parameter to remove "DOCUMENTO PROVISÓRIO" from cover page
\documentclass[11pt,a4paper,oneside]{memoir}

%%% This file is a template for the cover of your
%%%  thesis.
%%% Lines like this one, starting with %%% , are
%%%  instructions. Search them to know where and
%%%  how to edit this file. After editing the
%%%  document you can delete them.
%%% Items marked with OPTIONAL can be removed.
%%% Most of the lines you will have to change
%%%  already have examples to help you understand.
%%% The complete list of options and commands can
%%%  be found in uaThesisTemplate.sty

%%% Select your area of study and department from
%%%  the following lists and replace them in the
%%%  \usepackage declaration ([area,department]).
%%%
%%% | area        |   | department |
%%% ---------------   --------------
%%% | arts        |   | dao        |
%%% | sciences    |   | dbio       |
%%% | education   |   | dcspt      |
%%% | economy     |   | deca       |
%%% | engineering |   | degeit     |
%%% | humanities  |   | de         |
%%% | health      |   | deti       |
%%% |             |   | demac      |
%%% |             |   | decivil    |
%%% |             |   | dem        |
%%% |             |   | dfis        |
%%% |             |   | dlc        |
%%% |             |   | dmat       |
%%% |             |   | dq         |
%%% |             |   | dcm        |
%%% |             |   | dgeo        |
%%% ---------------   --------------
%% Import the UA thesis template
\usepackage[engineering,deti]{uaThesisTemplate}

%% Optional packages for utf8 encoding and T1 fonts
\usepackage[utf8]{inputenc}
\usepackage[T1]{fontenc}
\renewcommand\familydefault{\sfdefault}

%% We will use english and portuguese
\usepackage[english,portuguese]{babel}

\usepackage{xspace}% used by \sigla

%%% This package is only used to create a dummy
%%%  text for the abstract. You can remove the
%%%  following line in the final document.
%\usepackage{blindtext}


%%%%%%%%%%%%%%%%%%%%%%%%%%%%%%%%%%%%%%%%%%%%%%%%%%
%%       DEFINE THESIS COVER FIELDS
%%                  START
%%%%%%%%%%%%%%%%%%%%%%%%%%%%%%%%%%%%%%%%%%%%%%%%%%

%%% Edit thesis year
%% Thesis Year
\ThesisYear{\the\year}


%%% Uncomment one of the following options:
%\DocumentTypeThesis
%\DocumentTypeThesisProposal
\DocumentTypeDissertation

%%% Uncomment if you are a student of a MAP Doctoral program 
%%% http://www.map.edu.pt/
%%% | area                | identifier |
%%% | Biology of Plants   | bioplant   |
%%% | Computer Science    | i          |
%%% | Physics             | fis        |
%%% | Telecommunications  | tele       |

%\MAPProgram{i}

%%% Edit author's name
%% Author's full name
\Author{Primeiro Nome}{Último Nome}

%%% Edit thesis title in Portuguese
%% Thesis Mandatory Title in Portuguese
\TitlePT{Título da Tese em Português (MÁXIMO 130 CARACTERES)}

%%% OPTIONAL: you can write your thesis title in
%%%  other languages. You can repeat the following
%%%  line as many times as you want, or remove it
%%%  if you only want the portuguese title.
%% Optional Titles in additional languages.
\TitleEN{Thesis Title in English (MAX 130 CHARACTERS)}

%%% Edit your supervisor's information.
%%%
%%% Use this command for male supervisor
%%% \Supervisor{name}{category}
%%%
%%% Or this command for female supervisor
%%% \Supervisora{name}{category}
%%%
%%% name   : Supervisor's name
%%% categoy: One of the following list
%%%
%%% | category                  |
%%% -----------------------------
%%% | catedrático/a             |
%%% | catedrático/a convidado/a |
%%% | associado/a c/ agregação  |
%%% | associado/a               |
%%% | auxiliar                  |
%%% | auxiliar convidado        |
%%% -----------------------------
%% Supervisor information
\Supervisor{(nome do orientador)}{associado}

%%% OPTIONAL: you can add a second supervisor.
%%%
%%% Use this command for male supervisor
%%% \SupervisorMember{name}{category}
%%% 
%%% Or this command for female supervisor
%%% \SupervisoraMember{name}{category}
%%%  {[preposition: (at, in, no, da)] department}
%%%  {[preposition: (at, in, do, na)] school}
%%%  {[preposition: (and, e)] }
%%%
%%% name      : See above.
%%% category  : See above.
%%% department: Can be a department, faculty or
%%%             other; Choose the preposition
%%%             accordingly.
%%% school    : University/School name; Choose the
%%%             preposition accordingly.
%%% last      : Write {e } (Portuguese) or {and } (English)
%%%             for the last enumerated Supervisor; Or
%%%             nothing {} for all other supervisors 
%% Optional SecondSupervisor information
\SupervisorMember{(co-orientador)}{auxiliar convidado}{do Departamento de Matemática}{da Universidade de Aveiro}{}

%%% Repeat the above declaration for additional
%%%  Supervisor members as many times as necessary.
%% Additional Supervisor members
\SupervisoraMember{(co-orientadora)}{associada c/ agregação}{do Departamento de Biologia}{da Universidade de Aveiro}{}
\SupervisorMember{(co-orientador)}{auxiliar convidado}{do Departamento de Física}{da Universidade de Aveiro}{e }

%%% Edit your degree information.
%%%
%%% \Degree{title}{name}
%%%
%%% title: Degree title (Mestre or Doutor)
%%% name : Degree name
%% Degree title and name
\Degree{Mestre}{Engenharia de Computadores e Telemática}

%%% OPTIONAL: edit grant support texts where
%%%  appropriate. Maximum two.
%%%
%%% \GrantText{grant1}{grant2}
%% Optional Grant support texts
\GrantText{Texto Apoio financeiro do POCTI no âmbito do III Quadro Comunitário de Apoio.\\
          \newline
          (if applicable- Portuguese mandatory)}
          {Texto Apoio financeiro da FCT e do FSE no âmbito do III Quadro Comunitário de Apoio.\\
          \newline
          (if applicable- Portuguese mandatory)}

%%% Edit the Jury President information.
%% First Member will be the President of the Jury
\JuriMember{Prof. Doutor João Antunes da Silva}{professor associado da Universidade de Aveiro}

%%% Repeat the above declaration for additional
%%%  members of the jury as many times as necessary.
%% Additional members of the Jury
\JuriMember{Prof. Doutor João Antunes da Silva}{professor associado da Universidade de Aveiro}
\JuriMember{Prof. Doutor João Antunes da Silva}{professor associado da Universidade de Aveiro}
\JuriMember{Prof. Doutor João Antunes da Silva}{professor associado da Universidade de Aveiro}
\JuriMember{Prof. Doutor João Antunes da Silva}{professor associado da Universidade de Aveiro}
\JuriMember{Prof. Doutor João Antunes da Silva}{professor associado da Universidade de Aveiro}



%%% OPTIONAL: Add an acknowledgement text
%% Optional Acknowledgement text
\Acknowledgements{Agradeço toda a ajuda a todos os meus colegas e companheiros.}

%%% Edit the keywords and abstract in Portuguese
%%% Don't forget to remove the \blindtext[3]
%%%  declaration.
%%% Don't remove the \selectlanguage statement or you will end with incorrect hyphenation
%%% Don't use commands in the Abstract. Only the \selectlanguage{}, \texttt{}, and \\ commands were configured to be supported. You can configure additional commands in \pdfstringdefDisableCommands
%% Mandatory Keywords and Abstract in Portuguese
\Abstract{palavras-chave}{texto livro, arquitetura, história, construção, materiais de construção, saber tradicional.}
         {resumo}{\selectlanguage{portuguese}Um resumo é um pequeno apanhado de um trabalho mais longo (como uma tese, dissertação ou trabalho de pesquisa). O resumo relata de forma concisa os objetivos e resultados da sua pesquisa, para que os leitores saibam exatamente o que se aborda no seu documento.\\Embora a estrutura possa variar um pouco dependendo da sua área de estudo, o seu resumo deve descrever o propósito do seu trabalho, os métodos que você usou e as conclusões a que chegou.\\Uma maneira comum de estruturar um resumo é usar a estrutura IMRaD. Isso significa: Introdução, Métodos, Resultados, Discussão. Veja mais pormenores aqui:\\\texttt{https://www.scribbr.com/dissertation/abstract/}}

%%% OPTIONAL: you can write the keywords and
%%%  abstract in other languages. You can repeat
%%%  the following lines as many times as you want
%%%  or remove them if you only want portuguese.
%%% Don't remove the \selectlanguage statement or you will end with incorrect hyphenation.
%%% Don't use commands in the Abstract. Only the \selectlanguage{}, \texttt{}, and \\ commands were configured to be supported. You can configure additional commands in \pdfstringdefDisableCommands
%%% You can modify the language to any other supported by Latex
%% Optional Keywords and Abstract in additional languages.
\Abstract{keywords}{textbook, architecture, history, construction, construction materials, traditional knowledge.}
         {abstract}{\selectlanguage{english}An abstract is a short summary of a longer work (such as a thesis, dissertation or research paper).\\The abstract concisely reports the aims and outcomes of your research, so that readers know exactly what your paper is about.\\Although the structure may vary slightly depending on your discipline, your abstract should describe the purpose of your work, the methods you’ve used, and the conclusions you’ve drawn.\\One common way to structure your abstract is to use the IMRaD structure. This stands for: Introduction, Methods, Results, Discussion. Check for more details here:\\\texttt{https://www.scribbr.com/dissertation/abstract/}}


%%% OPTIONAL: Acknowledge the use of AI Tools
\AIUse{I acknowledge the use of [insert AI system(s) and link] to [specific use of generative artificial intelligence or other tasks]. I acknowledge the use of [software, codes or platforms] to [specific use software, codes or platforms or to other tasks].
\newline\newline
Example 1: I acknowledge the use of ChatGPT 3.5 (Open AI, https://chat.openai.com) to summarise the initial notes and to proofread the final draft and the use of Office365 (Microsoft, https://www.office.com) for text writing and productivity.
\newline\newline
Example 2: No content generated by AI technologies has been used in this Thesis.
}
%Reconheço o uso de [inserir sistema(s) de IA e respetiva hiperligação] para [indicar utilização específica de IA ou outras tarefas]. Reconheço a utilização de [indicar software, código ou plataforma] para [indicar utilização específica de software, código ou platafoma ou para outras tarefas].
%Exemplo 1: Reconheço a utilização do ChatGPT 3.5 (Open AI, https://chat.openai.com) para resumir as notas iniciais e para rever o rascunho final, a utilização do Office365 (Microsoft, https://www.office.com) para a escrita do texto.
%Exemplo 2: Não foram utilizados no presente trabalho quaisquer conteúdos gerados por tecnologias de IA.

%%%%%%%%%%%%%%%%%%%%%%%%%%%%%%%%%%%%%%%%%%%%%%%%%%
%%       DEFINE THESIS COVER FIELDS
%%                  END
%%%%%%%%%%%%%%%%%%%%%%%%%%%%%%%%%%%%%%%%%%%%%%%%%%

%%%%%%%%%%%%%%%%%%%%%%%%%%%%%%%%%%%%%%%%%%%%%%%%%%
%%       BEGIN DOCUMENT PACKAGES
%%%%%%%%%%%%%%%%%%%%%%%%%%%%%%%%%%%%%%%%%%%%%%%%%%
% Enable for for Overleaf support
%\usepackage{lmodern}

\usepackage{ifthen}
\def\useoverleaf{0}  % change to non-zero (for instance, 1) to enable it

\makeatletter
\newcommand{\makecoverfile}[0]{%
  \immediate\write18{latexmk -pdf cover.tex}%
}
\makeatother

% For PDF merging
\usepackage{pdfpages}

% Set DPI to 300
\pdfpxdimen=\dimexpr 1in/300\relax

% Allow the use of a larger number of packages
\usepackage{morewrites} 

% For English and Portuguese languages
% Portuguese will be the default.
% Uncomment \setlanguage below to change it
\usepackage[english,portuguese]{babel}

% Uncomment to use a custom date format
%\usepackage{datetime}
%\newdateformat{thesisdate}{\monthname[\THEMONTH] \THEYEAR} % Month Year

% Make pdf look better
\usepackage{microtype} 

% Uncomment to enable floats on facing pages
%\usepackage{dpfloat}

% Side by side figures
% Eg. Fig 1a, Fig 1b
\usepackage[hang,small,bf]{caption}
%\let\tion\undefined
%\let\subfloat\undefined
\usepackage{subcaption}

%\RequirePackage{textcase}

% Dropped Caps
%\usepackage{lettrine}

% Configure Hyperlink color
% As a matter or style, you may use this to enable/disable color boxes on links
%\usepackage[breaklinks=true,colorlinks=false,linkcolor=blue]{hyperref}
% Or use the default values provided by the hyperref package
\usepackage{hyperref}

% Disable commands not supported as PDF string
\pdfstringdefDisableCommands{%
  \def\\{ }%
  \def\texttt#1{#1}%
  \def\selectlanguage#1{}%
}

% Configure pdf information dictionary
\hypersetup{
  pdfauthor={{\ciAuthorFirst} {\ciAuthorLast}},
  pdftitle={\ciTitlePT},
  pdfkeywords={\ciKeywords},
  pdfsubject={\ciAbstractText}
}

% Redefine section names according to your preference
%\def\sectionautorefname{Section}
%\def\chapterautorefname{Chapter}
%\def\figureautorefname{Figure}
%\def\listingautorefname{Listing}
%\def\tableautorefname{Table}

% Redefine code boxes
\ifthenelse{\equal{\useoverleaf}{0}}
{\usepackage[outputdir=build]{minted}}
{\usepackage{minted}}%

\addto\captionsportuguese{%
  \renewcommand\listingscaption{Código}
}
\fvset{fontsize=\footnotesize} % Make Code blocks smaller than text
\usepackage{csquotes}

% Add support for PDF Comments
\usepackage{comment}
\ifthenelse{\equal{\useoverleaf}{0}}
{\usepackage{pdfcomment}}{}
\usepackage{bookmark} % New Bookmarks

% For Multiple columns in Glossary
\usepackage{multicol}

% Add support for Math symbols
\usepackage{amsmath}
\usepackage{amssymb}

% Add support for graphics
\usepackage{graphicx}

% Add support for Colors
\usepackage{xcolor}

% Add support for the Euro symbol
\usepackage{eurosym}

% Add support for missingfigure and todo
\usepackage{todonotes}

% Setup bibliography with Biber using IEEE style for proper UTF-8 support
\usepackage[backend=biber, style=ieee, sorting=none, natbib=true, mincitenames=1, maxcitenames=2]{biblatex}
\bibliography{bib/references.bib}

% Use acronyms
\usepackage[printonlyused]{acronym} % For acronyms

% Indenting the first paragraph after section start
\usepackage{indentfirst}

% For fixing listoflistings with memoir
\usepackage{xparse}

% Uncomment the next lines to enable chart support through pgf and tikz
% This may require you to install further packages in your Tex system
%\usepackage[version=0.96]{pgf}
%\usepackage{tikz}

% UML support
%\usepackage{pgf-umlsd}

% Trees, Arrows, Mindmaps and other popular objects
%\usetikzlibrary{arrows,shadows,trees,shapes,decorations,automata,backgrounds,petri,mindmap} % for pgf-umlsd

% Package to master SI units
\usepackage[detect-weight=true, binary-units=true]{siunitx}
% For Electric Circuits
%\sisetup{load-configurations = binary}

% Set Voltage direction accordingly
% Option : oldvoltagedirection,nooldvoltagedirection,RPvoltages,EFvoltages
% More information at: https://mirrors.ibiblio.org/CTAN/graphics/pgf/contrib/circuitikz/doc/circuitikzmanual.pdf
% By default this template is using the Old Voltage Direction
%\usepackage[oldvoltagedirection,american,cuteinductors,smartlabels]{circuitikz}
%\usetikzlibrary{calc}
%\ctikzset{bipoles/thickness=1}
%\ctikzset{bipoles/length=0.8cm}
%\ctikzset{bipoles/diode/height=.375}
%\ctikzset{bipoles/diode/width=.3}
%\ctikzset{tripoles/thyristor/height=.8}
%\ctikzset{tripoles/thyristor/width=1}
%\ctikzset{bipoles/vsourceam/height/.initial=.7}
%\ctikzset{bipoles/vsourceam/width/.initial=.7}
%\tikzstyle{every node}=[font=\small]
%\tikzstyle{every path}=[line width=0.8pt,line cap=round,line join=round]

% For inline TT text (e.g. code snippets)
\usepackage{verbatim}

% Frames around figures and allow force placement
\usepackage{float}

% Configure Float style
%\floatstyle{boxed}
%\restylefloat{table}
%\restylefloat{figure}
%\restylefloat{lstlisting}

% For test purposes you may use the lipsum package to create dummy text
\usepackage{lipsum} % REMOVE

%Keep floats inside section!
\usepackage[section]{placeins}
\let \oldsubsubsection \subsubsection
\renewcommand{\subsubsection}[2][]{
  \FloatBarrier
  \oldsubsubsection#1{#2}
}
\let \oldsubsection \subsection
\renewcommand{\subsection}[2][]{
  \FloatBarrier
  \oldsubsection#1{#2}
}
\let \oldsection \section
\renewcommand{\section}[2][]{
  \FloatBarrier
  \oldsection#1{#2}
}
\let \oldchapter \chapter
\renewcommand{\chapter}[2][]{
  \FloatBarrier
  \oldchapter#1{#2}
}

% Use the built-in division styling
\headstyles{memman}

% Include subsections in the TOC
\settocdepth{subsection}

% Numbering down to subsections as well
\setsecnumdepth{subsection}

% extra index for first lines
\makeindex[lines]

% Margins for University of Aveiro Thesis
\setlrmarginsandblock{3cm}{2.5cm}{*}
\setulmarginsandblock{3cm}{4cm}{*}
\setlength{\footskip}{20pt}
\checkandfixthelayout

% Or select your custom spacing to make any ajustment
%\addtolength{\parskip}{0.5\baselineskip}
\linespread{1.5}

\newcommand\mainmatterWithoutReset
{\edef\temppagenumber{\arabic{page}}%
  \mainmatter
  \setcounter{page}{\temppagenumber}%
}


%%%%%%%%%%%%%%%%%%%%%%%%%%%%%%%%%%%%%%%%%%%%%%%%%%
%%       BEGIN DOCUMENT
%%%%%%%%%%%%%%%%%%%%%%%%%%%%%%%%%%%%%%%%%%%%%%%%%%

\begin{document}

%%% By default we consider portuguese to be the
%%%  main language (the last one in the usepackage
%%%  declaration). If you're writing your thesis
%%%  in english uncomment the following line.
%%% Set the proper language or you will end with incorrect hyphenation!
%\selectlanguage{english}

%% COVER PAGES
\MakeCover

% Fix the numbering scheme by having a ghost style for page numbering
%\pagenumbering{Alph}

% Uncomment to enable English
%\selectlanguage{english}

% Front matter

%Custom Chapter style named `thesis`
\makechapterstyle{thesis}{% Based on ell
  \chapterstyle{default}
  \renewcommand*{\chapnumfont}{\normalfont\sffamily}
  \renewcommand*{\chaptitlefont}{\normalfont\Huge\sffamily}
  \settowidth{\chapindent}{\chapnumfont 111}
  \renewcommand*{\chapterheadstart}{\begingroup
    \vspace*{\beforechapskip}%
    \begin{adjustwidth}{}{-\chapindent}%
    \hrulefill
    \smash{\rule{0.4pt}{15mm}}
    \end{adjustwidth}\endgroup}
  \renewcommand*{\printchaptername}{}
  \renewcommand*{\chapternamenum}{}
  \renewcommand*{\printchapternum}{%
    \begin{adjustwidth}{}{-\chapindent}
    \hfill
    \raisebox{10mm}[0pt][0pt]{\fontsize{30}{25}\selectfont\chapnumfont \thechapter}%
                              \hspace*{1em}
    \end{adjustwidth}\vspace*{-3.0\onelineskip}}
  \renewcommand*{\printchaptertitle}[1]{%
    \vskip\onelineskip
    \raggedleft {\chaptitlefont ##1}\par\nobreak\vskip 4\onelineskip}}


% Select chapter style from existing or select custom
%\chapterstyle{thesis} % Others: dowding, demo2, dash, chappell, brotherton, bianchi, ger, madsen, tatcher, veelo,indexes)
% thesis can also be used as defined previously
% Check the memoir documentation for the available themes
% Default is veelo
\chapterstyle{veelo}
\makeoddfoot{plain}{}{\thepage}{} % Added by André Zúquete to fix a page numbering issue on the veelo chapter style

% Select Page style
\pagestyle{plain}

% If you feel adventurous you can also define all aspects of your theme
% Use either this input or the chapterstyle before
% % Rules
\newcommand{\thinRule}{\rule{\textwidth}{0.25pt}}

% Customize heading appearances
% Define styles
\newcommand{\partSize}{\Huge}
\newcommand{\partStyle}{\lsstyle\scshape}
\newcommand{\chapterSize}{\Huge}
\newcommand{\chapterStyle}{\lsstyle\scshape}
\newcommand{\chapterAfter}{}
\newcommand{\sectionSize}{\Large}
\newcommand{\sectionStyle}{\scshape\MakeTextLowercase}
\newcommand{\subsectionSize}{\large}
\newcommand{\subsectionStyle}{\scshape\MakeTextLowercase}
\newcommand{\subsubsectionSize}{\large}
\newcommand{\subsubsectionStyle}{\scshape\MakeTextLowercase}
\newlength{\partNumSizePt}
\setlength{\partNumSizePt}{60pt}
\newlength{\chapterNumSizePt}
\setlength{\chapterNumSizePt}{60pt}
\newcommand{\partNumSize}{%
  \fontsize{\partNumSizePt}{1.2\partNumSizePt}\selectfont%
}
\newcommand{\partNumStyle}{\partChapterNumColor}
\newcommand{\chapterNumSize}{%
  \fontsize{\chapterNumSizePt}{1.2\chapterNumSizePt}\selectfont%
}
\newcommand{\chapterNumStyle}{\partChapterNumColor}

% Customize parts
\renewcommand{\partnamefont}{\partSize\partStyle}
\renewcommand{\partnumfont}{\partNumSize\partNumStyle}
\renewcommand{\printpartname}{}
\renewcommand{\printparttitle}[1]{%
  \normalfont\normalcolor\partnamefont #1
}

% Customize chapters
\makeatletter
\setlength{\beforechapskip}{30pt}
\renewcommand*{\chapterheadstart}{\vspace*{\beforechapskip}}
\setlength{\afterchapskip}{3ex}
\setlength{\midchapskip}{3ex}
\renewcommand*{\chapnamefont}{%
  \Large\flushright\chapterStyle\partChapterNumColor%
}
\renewcommand*{\chapnumfont}{\chapterNumSize\chapterNumStyle}
\renewcommand*{\chaptitlefont}{%
  \normalfont\flushleft\normalcolor\chapterSize\chapterStyle%
}
\renewcommand*{\printchaptername}{%
  \chapnamefont\MakeTextLowercase{\@chapapp}%
}
\renewcommand*{\chapternamenum}{\quad}
\renewcommand*{\printchapternum}{%
%  \chapnumfont\textls[-75]{\classicstylenums{\thechapter}}%
 \chapnumfont\textls[-75]{\thechapter}%

}
\renewcommand*{\printchaptertitle}[1]{%
  \chaptitlefont #1
  \chapterAfter
}
\makeatother
% Customize sections and subsections
\setsecnumformat{\csname my#1\endcsname\quad}
\setsecheadstyle{\sectionSize\sectionStyle}
\newcommand{\mysection}{{\thesection}}
\setlength{\beforesecskip}{3em}


\setsubsecheadstyle{\subsectionSize\subsectionStyle}
\newcommand{\mysubsection}{{\normalfont\subsectionSize\thesubsection}}
\setlength{\beforesubsecskip}{3em}

\setsubsubsecheadstyle{\subsubsectionSize\subsubsectionStyle}
\newcommand{\mysubsubsection}{{\normalfont\subsubsectionSize\thesubsubsection}}
\setlength{\beforesubsubsecskip}{2em}

% Customize "Table of ..." appearance
% Customize headings
\newcommand{\renewPrintXTitle}[1]{%
  \renewcommand{#1}[1]{%
    \printchaptertitle{##1}%
  }%
}
\renewPrintXTitle{\printtoctitle}
\renewPrintXTitle{\printlottitle}
\renewPrintXTitle{\printloftitle}

% Customize ToC headings
\renewcommand{\cftpartfont}{\partChapterNumColor\partStyle}
\renewcommand{\cftchapterfont}{\chapterStyle}
\renewcommand{\cftsectionfont}{}
\renewcommand{\cftsubsectionfont}{}
\renewcommand{\cftfigurefont}{}
\renewcommand{\cfttablefont}{}
\newcommand{\cftlstlistingfont}{}

% Increase number width
\newlength{\cftNumWidthIncrease}
\setlength{\cftNumWidthIncrease}{0.25em}
\addtolength{\cftpartnumwidth}{\cftNumWidthIncrease}
\addtolength{\cftchapternumwidth}{\cftNumWidthIncrease}
\addtolength{\cftsectionindent}{\cftNumWidthIncrease}
\addtolength{\cftsubsectionindent}{\cftNumWidthIncrease}
% No leader dots
%\renewcommand*{\cftpartdotsep}{\cftnodots}
%\renewcommand*{\cftchapterdotsep}{\cftnodots}
%\renewcommand*{\cftsectiondotsep}{\cftnodots}
%\renewcommand*{\cftsubsectiondotsep}{\cftnodots}
%\renewcommand*{\cftfiguredotsep}{\cftnodots}
%\renewcommand*{\cfttabledotsep}{\cftnodots}
%\newcommand*{\cftlstlistingdotsep}{\cftnodots}
% Set page numbers immediately after entry text
\newcommand{\tocEntryPageSep}{\hspace{1em}}
\renewcommand{\cftpartleader}{\cftdotfill{\cftdotsep}}
%\renewcommand{\cftpartafterpnum}{\cftparfillskip}
%\renewcommand{\cftchapterleader}{\tocEntryPageSep}
\renewcommand{\cftchapterleader}{\cftdotfill{\cftdotsep}}
%\renewcommand{\cftchapterafterpnum}{\cftparfillskip}
\renewcommand{\cftsectionleader}{\cftdotfill{\cftdotsep}}
%\renewcommand{\cftsectionafterpnum}{\cftparfillskip}
\renewcommand{\cftsubsectionleader}{\cftdotfill{\cftdotsep}}
%\renewcommand{\cftsubsectionafterpnum}{\cftparfillskip}
\renewcommand{\cftfigureleader}{\cftdotfill{\cftdotsep}}
%\renewcommand{\cftfigureafterpnum}{\cftparfillskip}
\renewcommand{\cfttableleader}{\cftdotfill{\cftdotsep}}
%\renewcommand{\cfttableafterpnum}{\cftparfillskip}
\newcommand{\cftlstlistingleader}{\cftdotfill{\cftdotsep}}
%\newcommand{\cftlstlistingafterpnum}{\cftparfillskip}
% Customize page numbers
\newcommand{\tocPageStyle}{\tocPageColor}
\renewcommand{\cftpartpagefont}{\tocPageStyle}
\renewcommand{\cftchapterpagefont}{\tocPageStyle}
\renewcommand{\cftsectionpagefont}{\tocPageStyle}
\renewcommand{\cftsubsectionpagefont}{\tocPageStyle}
\renewcommand{\cftfigurepagefont}{\tocPageStyle}
\renewcommand{\cfttablepagefont}{\tocPageStyle}
\newcommand{\cftlstlistingpagefont}{\tocPageStyle}

% Abstract
% Remove indents around abstract text
\setlength{\absleftindent}{0pt}
\setlength{\absrightindent}{0pt}
% Change font size to conform with the rest of the document text
\renewcommand{\abstracttextfont}{\normalsize}

% Customize headers and footers including page numbers
\newcommand{\hfTextSize}{\footnotesize}
\newcommand{\headTextStyle}{\lsstyle\scshape\MakeTextLowercase}
\nouppercaseheads
\makeevenhead{headings}%
             {\hfTextSize\thepage}%
             {}%
             {\hfTextSize\headTextStyle\leftmark}
\makeevenhead{plain}%
             {\hfTextSize\thepage}%
             {}%
             {\hfTextSize\headTextStyle\leftmark}
\makeoddhead{headings}%
            {\hfTextSize\headTextStyle\rightmark}%
            {}%
            {\hfTextSize\thepage}
\makeoddhead{plain}%
            {\hfTextSize\headTextStyle\rightmark}%
            {}%
            {\hfTextSize\thepage}


% Customize captions
\newcommand{\captionSize}{\small}
\newcommand{\captionStyle}{\scshape}
\newcommand{\captionWidthRatio}{0.9}

\captionnamefont{\captionSize\captionStyle}
\captiontitlefont{\captionSize}
\captiondelim{ -- }
\captiontitlefinal{}
\changecaptionwidth
%\captionwidth{\captionWidthRatio\textwidth}

% Define colors
%\newcommand{\titleColor}{\color[rgb]{0.616, 0.0627, 0.176}}
\newcommand{\titleColor}{\color[rgb]{0,0,0}}

\newcommand{\partChapterNumColor}{\titleColor}
\newcommand{\dropCapColor}{\titleColor}
%\newcommand{\tocPageColor}{\color[rgb]{0.0980, 0.329, 0.651}}

\newcommand{\tocPageColor}{\color[rgb]{0, 0,0}}
\definecolor{shade0}{rgb}{1.0 , 1.0 , 1.0 }
\definecolor{shade1}{rgb}{0.9 , 0.9 , 0.9 }
\definecolor{shade2}{rgb}{0.8 , 0.8 , 0.8 }
\definecolor{shade3}{rgb}{0.65, 0.65, 0.65}
\definecolor{shade4}{rgb}{0.45, 0.45, 0.45}
\definecolor{shade5}{rgb}{0.0 , 0.0 , 0.0 }



%Exclude sub figures from List of Figures
%\captionsetup[subfloat]{list=no}

% Texts
\newenvironment{introduction}
{%
  \begin{minipage}{\textwidth}%
   \itshape%
}
{%
  \end{minipage}%
  \par\addvspace{2\baselineskip plus 0.2\baselineskip minus 0.2\baselineskip}%
}

\frontmatter

\tightlists
\midsloppy
\raggedbottom

\setcounter{tocdepth}{2} %subsections are added to the TOC
\setcounter{secnumdepth}{4} %subsubsections are numbered


\IndexContents
\IndexTables
\IndexFigures

%\renewcommand{\listingscaption}{Código}
%\renewcommand{\listoflistingscaption}{Lista de Excertos de Código}
\IndexListings

\RenewDocumentCommand \chapter { s O{#3} m }{%
  \FloatBarrier
  \IfValueTF{#1}  % if optional star is seen
    {\oldchapter*{#2}}
    {\oldchapter#1{#2}}
}



% Reset Chapters
\renewcommand{\chapter}[2][]{
  \FloatBarrier
  \oldchapter#1{#2}
}

% Print Glossary
\clearpage
{\small\chapter{References}

\footnotesize
\SingleSpacing

\begin{multicols}{2}
\begin{acronym}[AAAAAA]

	\acro{h2o}[H2O]{Water}
	\acro{adsl}[ADSL]{Asymmetric Digital Subscriber Line\cite{adsl}}


\end{acronym}
\end{multicols}

}

%%%%%%%%%%%%%%%%%%%%%%%%%%%%%%%%%%%%%%%%%%%%%%%%%%%%%%%
% Main document starts here
%%%%%%%%%%%%%%%%%%%%%%%%%%%%%%%%%%%%%%%%%%%%%%%%%%%%%%%

\mainmatter



\begingroup
% Line spacing: 1.5 pt 
\OnehalfSpacing

\fontfamily{lmss}\selectfont

%%%%%%%%%%%%%%%%%%%%%%%%%%%%%%%%%%%%%%%%%%%%%%%%%%%%%%%
% Start of Thesis text 
%%%%%%%%%%%%%%%%%%%%%%%%%%%%%%%%%%%%%%%%%%%%%%%%%%%%%%%

%% CHAPTERS


\chapter{Introdução}
\label{chapter:introduction}

\begin{introduction}
A short description of the chapter.

A memorable quote can also be used.
\end{introduction}



\section{Acrónimos}

Primeira e seguintes referências: \ac{h2o}, \ac{h2o}

Plural, acrónimo expandido e curto: \acp{h2o}, \acs{h2o}, \acl{h2o}

Com citação\footnote{Necessária entrada na bibliografia}: \ac{adsl}, \ac{adsl}


\section{Fontes}

\begin{itemize}
\item{\tiny Tiny}
\item{\scriptsize Scriptsize}
\item{\footnotesize Footnotes}
\item{\small Small}
\item{\normalsize Normal}
\item{\large large}
\item{\Large Large}
\item{\LARGE LARGE}
\item{\huge huge}
\item{\Huge Huge}
\end{itemize}

\section{Unidades}

Utilizando o pacote \verb|siunitx| é possível utilizar unidades do Sistema Internacional. Exemplo: a aceleração da gravidade é de \SI{9.8}{\metre\per\second\squared} e um ficheiro ocupa \SI{1}{\mebi\byte}. 

\section{Code Blocks}
%\lipsum[5]
Uma listagem pode ser apresentada com o ambiente \texttt{listing}, que é um float (objeto flutuante, tal como uma figura ou uma tabela).

A listagem em Código~\ref{lbl:snippet-test} mostra um exemplo em C.

\begin{listing}[h]
\begin{minted}{c}

#include <stdio.h>
#define N 10
/* Block
 * comment */
 
int main()
{
    int i;
 
    // Line comment.
    puts("Hello world!");
 
    for (i = 0; i < N; i++)
    {
        puts("LaTeX is also great for programmers!");
    }
 
    return 0;
}
\end{minted}
\caption{This caption appears below the code.}
\label{lbl:snippet-test}
\end{listing}

%\lipsum[5]

\section{Citações}

Algumas formas distintas de citar:

\begin{itemize}
    \item \textbf{Apenas referência}:~\cite{rfc44}
    \item \textbf{Apenas data}:~\citedate{rfc44}
    \item \textbf{Apenas ano}:~\citeyear{rfc44}
    \item \textbf{Apenas autor}:~\citeauthor{rfc44}
    \item \textbf{Apenas editor}:\citelist{rfc44}{organization}
    \item \textbf{Autor e referência}:\citet{rfc44}
\end{itemize}

%\chapter{Capítulo de Demonstração}
\label{chap:demonstration}

\begin{introduction}
Este capítulo foi criado por um modelo de Inteligência Artificial para demonstrar as várias funcionalidades disponíveis neste template de \LaTeX{}. 
\end{introduction}

O capítulo foi concebido para ser um guia prático para estudantes, apresentando elementos comuns de uma tese, como formatação de texto, citações, figuras, tabelas e listagens de código. O objetivo é fornecer exemplos claros que possam ser facilmente adaptados para o seu próprio trabalho.

\section{Estrutura do Documento}
Uma tese é tipicamente estruturada em capítulos, secções e subsecções. Isto ajuda a organizar o conteúdo de forma lógica e facilita a navegação para os leitores.

\subsection{Secções e Subsecções}
Pode criar secções usando o comando \verb|\section{}| e subsecções com \verb|\subsection{}|. Para um aninhamento mais profundo, \verb|\subsubsection{}| também está disponível. O template tratará automaticamente da numeração e adicioná-los-á ao índice.

\subsubsection{Uma Nota sobre Rótulos (Labels)}
É uma boa prática adicionar um \verb|\label{}| após cada capítulo, secção, figura e tabela. Isto permite referenciá-los no texto usando \verb|\ref{}| ou \verb|\autoref{}|, que gera automaticamente o tipo de referência correto (por exemplo, "Capítulo 1", "Figura 2.3"). Por exemplo, este capítulo é o Capítulo~\ref{chap:demonstration}.

\section{Citações e Bibliografia}
Citar as suas fontes corretamente é crucial. Este template usa \texttt{biblatex} para a gestão da bibliografia. Pode adicionar citações de várias maneiras. Por exemplo, pode citar uma fonte entre parênteses, como as normas da Universidade de Aveiro~\cite{modelos}.

Aqui estão alguns comandos de citação comuns:
\begin{itemize}
    \item \verb|\cite{chave}|: Para uma citação parentética, e.g.,~\cite{modelos}.
    \item \verb|\citet{chave}|: Para uma citação textual onde o autor faz parte do texto.
    \item \verb|\citeauthor{chave}|: Para obter apenas o(s) autor(es).
    \item \verb|\citeyear{chave}|: Para obter apenas o ano.
\end{itemize}

\section{Figuras e Tabelas}

\subsection{Inclusão de Figuras}
As figuras são essenciais para visualizar dados e conceitos. Pode incluir imagens usando o ambiente \texttt{figure} e o comando \verb|\includegraphics|. A Figura~\ref{fig:placeholder} mostra um exemplo. Lembre-se de usar imagens de alta qualidade e fornecer uma legenda descritiva.

\begin{figure}[H]
    \centering
    \missingfigure[figwidth=0.6\textwidth]{Uma figura de exemplo}
    \caption{Um exemplo de uma figura provisória. O pacote \texttt{todonotes} fornece o comando \texttt{missingfigure} para este propósito.}
    \label{fig:placeholder}
\end{figure}

\subsection{Criação de Tabelas}
As tabelas são usadas para apresentar dados estruturados. O ambiente \texttt{table} é um flutuante, semelhante a \texttt{figure}. Dentro dele, o ambiente \texttt{tabular} é usado para construir a tabela. A Tabela~\ref{tab:example} é um exemplo simples.

\begin{table}[H]
    \centering
    \caption{Uma tabela de exemplo mostrando diferentes tipos de dados.}
    \label{tab:example}
    \begin{tabular}{l c c}
        \hline
        \textbf{Parâmetro} & \textbf{Valor} & \textbf{Unidade} \\
        \hline
        Velocidade da Luz (c) & $2.998 \times 10^8$ & \si{\metre\per\second} \\
        Constante Gravitacional (G) & $6.674 \times 10^{-11}$ & \si{\newton\metre\squared\per\kilogram\squared} \\
        Tamanho de um ficheiro & 1 & \si{\mebi\byte} \\
        \hline
    \end{tabular}
\end{table}

\section{Listagens de Código}
Este template usa o pacote \texttt{minted} para formatar código, que oferece um excelente realce de sintaxe. Pode incluir blocos de código num ambiente flutuante \texttt{listing}, como mostrado no Código~\ref{lst:c-example}.

\begin{listing}[H]
\begin{minted}[frame=lines,
               framesep=2mm,
               baselinestretch=1.2,
               fontsize=\footnotesize,
               linenos]{c}
#include <stdio.h>

int main()
{
    // Imprime uma saudação na consola
    printf("Olá, Mundo!\n");
    return 0;
}
\end{minted}
\caption{Um programa "Olá, Mundo!" escrito em C.}
\label{lst:c-example}
\end{listing}

Para código em linha, pode usar \verb|\mintinline{language}{code}|, por exemplo: \\ \mintinline{python}{print("Olá de Python!")}.

\section{Fórmulas Matemáticas}
O pacote \texttt{amsmath} fornece ferramentas abrangentes para a formatação de matemática. Pode escrever matemática em linha como $E = mc^2$ envolvendo-a em cifrões. Para equações em destaque, use o ambiente \texttt{equation}, que também fornece numeração.

\begin{equation}
    \int_0^\infty e^{-x^2} dx = \frac{\sqrt{\pi}}{2}
    \label{eq:gaussian}
\end{equation}

A Equação~\ref{eq:gaussian} é a famosa integral Gaussiana.

\section{Acrónimos e Unidades}

\subsection{Acrónimos}
O pacote \texttt{acronym} ajuda a gerir acrónimos. A primeira vez que usa um acrónimo, como \ac{adsl}, ele será expandido por extenso. As utilizações subsequentes, como \ac{adsl}, mostrarão apenas a forma abreviada. Também pode forçar a forma longa com \verb|\acl{h2o}| (\acl{h2o}) ou a forma curta com \verb|\acs{h2o}| (\acs{h2o}). Os plurais são tratados com \verb|\acp{h2o}|, que produz \acp{h2o}.


\subsection{Unidades SI}
O pacote \texttt{siunitx} deve ser usado para todas as grandezas físicas para garantir uma formatação correta e consistente. Por exemplo, a aceleração da gravidade é aproximadamente \SI{9.8}{\metre\per\second\squared}. Os tamanhos de dados também podem ser representados, por exemplo, \SI{1024}{\kibi\byte} é igual a \SI{1}{\mebi\byte}.
  % uncomment to add further chapters
%\input{chapters/chapter3}
%\input{chapters/chapter4}

\endgroup

\backmatter

%%%%%%%%%%%%%%%%%%%%%%%%%%%%%%%%%%%%%%%%%%%%%%%%%%%%%%%
% Print all used references
%%%%%%%%%%%%%%%%%%%%%%%%%%%%%%%%%%%%%%%%%%%%%%%%%%%%%%%

\begingroup
\renewcommand{\bibfont}{\footnotesize}
% Redefine References name to Portuguese
% Change if you are using english
\defbibheading{bibliography}[Referências]{
	\chapter{#1}
}
\SingleSpacing
\setlength\bibitemsep{8pt}
\printbibliography[heading=bibliography]
\endgroup


%%%%%%%%%%%%%%%%%%%%%%%%%%%%%%%%%%%%%%%%%%%%%%%%%%%%%%%
% Load appendix
%%%%%%%%%%%%%%%%%%%%%%%%%%%%%%%%%%%%%%%%%%%%%%%%%%%%%%%

\mainmatterWithoutReset
\appendix

\chapter{Informações Adicionais}
\label{chap:appendix-more}

\begin{introduction}
Este apêndice foi criado para fornecer informações adicionais que complementam o conteúdo principal da tese.
\end{introduction}

\section{Seção de Exemplo}
Este é um exemplo de seção adicional que pode ser incluída em um apêndice. Você pode adicionar mais seções conforme necessário para fornecer detalhes suplementares, dados brutos, ou qualquer outra informação relevante que não foi incluída no corpo principal do trabalho.

\subsection{Outra Seção de Exemplo}
Aqui está outra seção de exemplo. Apêndices são úteis para manter o fluxo do texto principal, enquanto ainda fornecem acesso a informações importantes para leitores interessados.





\end{document}
