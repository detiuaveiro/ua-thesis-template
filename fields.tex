%%% Define ALL the variables in this file. After that, follow the 
%%%  instructions of cover.tex file where you can decide if you will use 
%%%  the titles and keywords in English/other language.


%%% Edit author's first name
%% Author's first name
\newcommand{\gAuthorFirst}{FIRST NAME}

%%% Edit author's last name
%% Author's last name
\newcommand{\gAuthorLast}{LAST NAME}

%%% Edit thesis title in Portuguese
%% Thesis Mandatory Title in Portuguese
\newcommand{\gTitle}{TÍTULO DA TESE (MÁXIMO 70 CARACTERES)}

%%% Edit the keywords in Portuguese
%% Mandatory Keywords in Portuguese
\newcommand{\gKeywords}{texto livro, arquitetura, história, construção, materiais de construção, saber tradicional.}

%%% Edit the abstract in Portuguese
%%% Don't forget to remove the \blindtext[3]
%%%  declaration.
%%% Don't remove the \selectlanguage statement or you will end with incorrect hyphenation
%% Mandatory Abstract in Portuguese
\newcommand{\gAbstract}{\selectlanguage{portuguese}Um resumo é um pequeno apanhado de um trabalho mais longo (como uma tese, dissertação ou trabalho de pesquisa). O resumo relata de forma concisa os objetivos e resultados da sua pesquisa, para que os leitores saibam exatamente o que se aborda no seu documento.\\Embora a estrutura possa variar um pouco dependendo da sua área de estudo, o seu resumo deve descrever o propósito do seu trabalho, os métodos que você usou e as conclusões a que chegou.\\Uma maneira comum de estruturar um resumo é usar a estrutura IMRaD. Isso significa: \begin{itemize} \item Introdução \item Métodos \item Resultados \item Discussão \end{itemize} Veja mais pormenores aqui:\\\texttt{https://www.scribbr.com/dissertation/abstract/}}
